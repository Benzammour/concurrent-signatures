\begin{frame}
	\frametitle{Security Model}

  \begin{definition}
    \begin{itemize}[<+->]
      \item formalizes security properties for a given system
      \item defines the adversaries power in said system
      \item if all properties are given, system is called \textit{\textcolor{Red}{secure}}
    \end{itemize}
  \end{definition}

  \visible<4->{we formalize security through}
  \begin{itemize}[<5->]
    \item Fairness
    \item Unforgeability
    \item Ambiguity
	\end{itemize}

\end{frame}

\begin{frame}
	\frametitle{Security Model}

	\begin{itemize}[<+->]
		\setlength\itemsep{1em}
    \item consider game with
      \begin{itemize}[<1->]
        \item \textcolor{Red}{adversary $E$}
        \item \textcolor{ForestGreen}{challenger $C$}
      \end{itemize}
		\item \textcolor{ForestGreen}{$C$} runs \setup and publishes all public keys
		\item \textcolor{Red}{$E$} can request\\[.2cm]
			$\left.\parbox{.5\textwidth}{%
			\begin{itemize}[<+->]
				\item \textbf{KGen\ldots}
				\item \textbf{KReveal\ldots}
				\item \textbf{ASign\ldots} 
				\item \textbf{AVerify / Verify\ldots}
				\item \textbf{Private Key Extraction\ldots}
			\end{itemize}%
			}\right\}\text{\ldots Queries from } C$
		\end{itemize}
\end{frame}