\begin{frame}
	\frametitle{Concrete Scheme}

	\begin{itemize}[<+->]
		\item based on Ring Signatures
			\begin{itemize}
				\item in turn based on: Schnorr Signatures
			\end{itemize}
		\item \setup 
		\begin{itemize}
			\item outputs two large primes $p$ and $q$ with $q \mid p-1$
			\item $p$, $q$ published with $g\in(\ints/p\ints)^\ast$
			\item $\sspace \equiv \kfspace = \ints_q$ and $\mespace \equiv \kspace = \{0,1\}^\ast$
			\item $H_1$, $H_2: \{0,1\}^\ast \to \ints_q$ cryptographic hash functions 
			\item $\keygen \coloneqq H_1$
			\item private keys chosen uniformly from $\ints_q$
			\item public keys $X_i = g^{x_i} \bmod p$ published
		\end{itemize}
	\end{itemize}
\end{frame}

\begin{frame}
	\frametitle{Lemmas}

	\begin{lemma}[Fairness]
		The concurrent signature scheme in our example is fair in the random oracle model
	\end{lemma}

	\begin{lemma}[Ambiguity]
		The concurrent signature scheme in our example is ambiguous in the random oracle model
	\end{lemma}	
\end{frame}

\begin{frame}
	\frametitle{Lemmas}

	\begin{lemma}[Unforgeability]
		The concurrent signature scheme in our example is existentially unforgeable under a chosen message attack in the random oracle model, assuming the hardness of the discrete logarithm problem
	\end{lemma}	
	\begin{proof}
		\textit{Rough Idea:} Assume our scheme is forgeable with an non-neglible probability. 
		Then, we construct an algorithm that solves the discrete logarithm problem.
		\textcolor{red}{using: forking lemma, non-neglible probabilities}
	\end{proof}
\end{frame}