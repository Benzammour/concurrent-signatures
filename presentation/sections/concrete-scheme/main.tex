\begin{frame}
	\frametitle{Concrete Scheme}

	\begin{itemize}[<+->]
		\item based on Ring Signatures
		\item \setup 
      \begin{itemize}
        \setlength\itemsep{.5em}
        \item outputs two large primes $p$ and $q$ with $q \mid p-1$, and $g\in(\ints/p\ints)^\ast$
        \item $\sspace \equiv \kfspace = \ints_q$ and $\mespace \equiv \kspace = \{0,1\}^\ast$
        \item $H_1$, $H_2: \{0,1\}^\ast \to \ints_q$ cryptographic hash functions 
        \item $\keygen \coloneqq H_1$
        \item public keys $X_i = g^{x_i} \bmod p$ published
      \end{itemize}
	\end{itemize}
\end{frame}

\begin{frame}
	\frametitle{Concrete Scheme}

  \begin{itemize}[<+->]
		\setlength\itemsep{.5em}
		\item \asign: $\goedel{\textcolor{SeaGreen}{X_i}, \textcolor{MidnightBlue}{X_j}, \textcolor{Mahogany}{x_i}, \textcolor{BurntOrange}{f}, M}$
    \begin{itemize}[<+->]
		  \setlength\itemsep{.5em}
			\item $h = H_2(g^{\textcolor{Plum}{t}} \textcolor{MidnightBlue}{X_j}^{\textcolor{BurntOrange}{f}}\bmod p ~\|~ M)$ with $\textcolor{Plum}{t\in\ints_q}$ random
			\item $h_1 = h - \textcolor{BurntOrange}{f} \bmod q$
			\item $s = \textcolor{Plum}{t} - h_1 \textcolor{Mahogany}{x_i} \bmod q$
		\end{itemize}
		\item \averify: $\goedel{\goedel{s, h_1, f}, \textcolor{SeaGreen}{X_i}, \textcolor{MidnightBlue}{X_j}, M}$
		\begin{itemize}[<+->]
			\item returns: $h_1 + f \overset{?}{=} H_2(g^s \textcolor{SeaGreen}{X_i}^{h_1} \textcolor{MidnightBlue}{X_j}^{\textcolor{BurntOrange}{f}} \bmod p ~\|~ M) \bmod q$
		\end{itemize}
	\end{itemize}
\end{frame}

\begin{frame}
	\frametitle{Lemmas}

	\begin{lemma}[Fairness]
		The concurrent signature scheme in our example is fair in the random oracle model
	\end{lemma}

	\begin{lemma}[Ambiguity]
		The concurrent signature scheme in our example is ambiguous in the random oracle model
  \end{lemma}	
  
  \begin{alertblock}{Lemma (Unforgeability)}
    The concurrent signature scheme in our example is existentially \textit{unforgeable} under a \textit{chosen message attack} in the random oracle model, assuming the hardness of the discrete logarithm problem
  \end{alertblock}	
\end{frame}
