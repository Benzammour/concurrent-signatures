\begin{frame}
	\frametitle{Discrete Logarithmic Problem}

	\begin{definition}[discrete logarithm]<1->
		\begin{itemize}
			\item Group $\mathbb{G}$ given
			\item $b^k$ always defined in $\mathbb{G}$ through $b^k = \underbrace{b\cdot b \cdots b}_{k\text{ times}}$
			\item the \textbf{discrete logarithm} is an integer $k$, such that $b^k = a$
		\end{itemize}
	\end{definition}
	%  
	\begin{definition}[group generator]<2->
		Let $\mathbb{G}$ be a cyclic group of order $p$. Then $g\in\mathbb{G}$ is generator of $\mathbb{G}$ if
			$$\mathbb{G} = \{g^i \bmod p \mid i \in \nats \}$$
	\end{definition}
\end{frame}

% \begin{frame}
% 	\frametitle{Discrete Logarithmic Problem}
% 	\begin{block}{Discrete-Logarithm Experiment $DLog_{\mathcal{A}, \mathcal{G}}(n)$}<1->
% 		\begin{itemize}
% 			\item Run $\mathcal{G}(1^n)$ to acquire $(\mathbb{G}, p, g)$ where $\mathbb{G}$ is a cyclic group, $p$ is its order (with $||p|| = n$) and $g\in \mathbb{G}$ is $\mathbb{G}$'s generator
% 			\item Choose a random $h\in\mathbb{G}$
% 			\item $\mathcal{A}$ is given $\mathbb{G}, p, g$ and $h$, and outputs $i\in \ints_p$
% 			\item The result of the experiment is $1$ if $g^i = h$, and $0$ otherwise.
% 		\end{itemize}
% 	\end{block}
% 	\begin{definition}<2->
% 		We say the \textbf{discrete logarithm problem is hard relative to} $\mathcal{G}$ if for all polynomial-time algorithms $\mathcal{A}$ there exists a negligible function $\textsf{negl}$ such that 
% 			$$\Pr[DLog_{\mathcal{A}, \mathcal{G}}(n) = 1] \leq \textsf{negl}(n)$$
% 	\end{definition}
% \end{frame}