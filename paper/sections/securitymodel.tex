To label our concurrent signature scheme as \textit{secure} a certain set of conditions has to hold. 
In this scheme, we define these properties to be fairness, ambiguity, and unforgeability.

To define the mentioned properties we, first of all, have to define our game, in which we have an adversary \(E\) and a challenger \(C\).
In this game \(C\) runs \textbf{SETUP} and publishes all public variables to the public domain. 

The adversary can query the challenger for certain information:
\begin{itemize}
  \item The first being a \textbf{KGen} query, meaning \(E\) can request \(C\) to generate a keystone-fix
  \item It can also request the challenger to release the keystone to a given keystone-fix through a \textbf{KReveal} query.
  \item \textbf{ASign} queries achieve the goal of receiving an \textit{ambiguous signature} from the challenger on an adequate input. 
  \item Lastly, the adversary can provide a public key for a \textbf{Private Key Extraction} query in order to receive the respective private key from \(C\).
\end{itemize}

Note, that \(C\) does not answer to \textbf{AVerify / Verify} queries because the adversary can run the needed algorithms itself.

\subsection{Fairness}
  In order to define fairness in our scheme, consider the previously defined game between the adversary and the challenger.
  The adversary can initially query everything it wants and \(C\) answers accordingly.
  Eventually, however, the adversary returns two public keys \(X_c, X_d\), a keystone \(k\in\kspace\), \(S=\goedel{\sigma, X_c, X_d, M}\) with \(\sigma=\goedel{s, h_1, f}\), and \(\textbf{AVERIFY}(S) = accept\).
  The adversary \(E\) wins if either
    \begin{enumerate}
      \item \(f\) was the output from a previous query, no \textbf{KReveal} query was made on \(f\) and if \(\goedel{k,S}\) is accepted by \textbf{VERIFY}, or
      \item If \(E\) produces a second \(S' = \goedel{\sigma', X_d, X_c, M'}\) with \(\sigma' = \goedel{s', h_1', f}\) such that \(\textbf{AVERIFY}(S')=accept\).
            Continuing, \(\goedel{k,S}\) is accepted by \textbf{VERIFY}, but  \(\goedel{k,S'}\) is rejected.
    \end{enumerate}
  In the first case, the adversary produces the keystone matching to an arbitrary \(f\).
  Therefore, \(E\) is able to force \(\textbf{VERIFY}(k, S) = \textit{accept}\) for any valid \(S\).
  The second case describes the scenario in which the adversary is able to create a signature, with the same keystone-fix, that is not bound upon releasing the keystone.

\subsection{Ambiguity}
  Our defined game is split into three phases, the first query phase, the challenge phase and, lastly, the second query phase. 
  After the challenger sent all public variables to the adversary, the adversary starts the first query phase by requesting anything it wants.
  Eventually, it has to send a, so-called, \textit{challenge tuple} to the challenger.
  This tuple is of the form \(\goedel{X_i, X_j, M}\) and represents the information the adversary wants to be challenged on.
  Then, the challenger flips a coin \(b\in\Set{0,1}\) and if \(b = 0\) it sends an ambiguous signature \(\sigma\) to the adversary on behalf of \(X_i\).
  Otherwise, the challenger sends an ambiguous signature on behalf of \(X_j\). 
  Now, the adversary can start its second query phase, and sooner or later sends the challenge bit, which is a guess whose signature the challenger sent to our adversary.
  The adversary wins if the challenge bit matches the flipped coin of the challenger.
  Meaning, if the adversary was able to guess whose signature was sent, it managed to circumvent the ambiguity of the scheme.

\subsection{Unforgeability}\label{unforgeability}
  We, again, consider the previously defined adversary--challenger game.
  The adversary can query everything it wants, but finally has to return \(S=\goedel{\sigma, X_c, X_d, M}\) where \(\sigma=\goedel{s, h_1, f}\), \(X_c, X_d\) being public keys and \(M\in\mespace\) being the message.
  Here the adversary wins the game if \(\textbf{AVERIFY}(S) = accept\) and if either
    \begin{enumerate}
      \item No \textbf{ASign} query was made with input \(\goedel{X_c, X_d, f, M}\) or \(\goedel{X_d, X_c, h_1, M}\).
            And no \textbf{Private Key Extraction} query was made on either \(X_c\) or \(X_d\), or
      \item No \textbf{ASign} query was made with input \(\goedel{X_c, X_i, f, M}\) for all \(X_i \neq X_c, X_i\in\uspace\) and no \textbf{Private Key Extraction} query was made for \(X_c\).
    \end{enumerate}
  Semantically, the first case means that the adversary has no information about any of the public keys and tries to forge a signature for \(X_c\) and/or \(X_d\).
  The second case, however, implies that the adversary has one private key (or at least enough information to use it) and tries to forge a signature on behalf of \(X_c\).
  This would imply that one of the parties wants to cheat on the other.

\subsection{Security of our scheme}\label{concretesec}
  Now that we defined our (concrete) concurrent signature scheme as well as when it can be called secure, we want to actually prove that it is.
  As mentioned before, this paper focuses on the unforgeability portion of the security proof.
  If one is interested in how the other properties are proven, reference \cite{chen2004concurrent}.\\
  Knowing all of this, we define
  \begin{lemma}\label{unforgeablelemma}
    The concrete concurrent signature scheme defined is existentially unforgeable under a chosen message attack in the random oracle model, assuming the hardness of the discrete logarithm problem.
  \end{lemma}