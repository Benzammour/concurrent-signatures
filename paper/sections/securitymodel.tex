To label our concurrent signature scheme as \textit{secure} a certain set of conditions has to hold. 
  In this scheme, we define these properties through fairness, ambiguity, and unforgeability.
  
  To define the mentioned properties we, first of all, have to define our game, in which we have an adversary \(E\) and a challenger \(C\).
  In this game \(C\) runs \textbf{SETUP} and publishes all public variables to the public domain. 
  
  The adversary can query the challenger for certain information. 
  First being a \textbf{KGen} query, meaning \(E\) can request \(C\) to generate a keystone-fix.
  It can also request \(C\) to release the keystone to a given keystone-fix through a \textbf{KReveal} query.
  The goal of \textbf{ASign} queries is to receive an \textit{ambiguous signature} from \(C\). 
  Lastly, the adversary can provide a public key for a \textbf{Private Key Extraction} query in order to receive the respective private key from \(C\).
  Note, that \(C\) does not answer to \textbf{AVerify / Verify} queries because the adversary can run the needed algorithms itself.

  \subsection{Fairness}
    In order to define fairness in our scheme, consider the previously defined game between the adversary and the challenger.
    The adversary can initially query everything it wants and \(C\) answers accordingly.
    Eventually, however, the adversary returns two public keys \(X_c, X_d\), a keystone \(k\in\kspace\), \(S=\goedel{\sigma, X_c, X_d, M}\) with \(\sigma=\goedel{s, h_1, f}\), and \(\textbf{AVERIFY}(S) = accept\).
    The adversary \(E\) wins if either
      \begin{enumerate}
        \item if \(f\) from previous query, no \textbf{KReveal} query was made on \(f\) and if \(\goedel{k,S}\) is accepted by \textbf{VERIFY}, or
        \item if \(E\) produces a second \(S' = \goedel{\sigma', X_d, X_c, M'}\) with \(\sigma' = \goedel{s', h_1', f}\) such that \(\textbf{AVERIFY}(S')=accept\).
              Continuing, \(\goedel{k,S}\) is accepted by \textbf{VERIFY}, but  \(\goedel{k,S'}\) is rejected.
      \end{enumerate}
    In the first case, the adversary produces the keystone matching to an arbitrary \(f\) which is valid.
    The second case describes the case in which the adversary is able to create a signature, with the same keystone-fix, that is not bound upon releasing the keystone.

  \subsection{Ambiguity}
    Our defined game is split into three phases now, the first query phase, the challenge phase and, lastly, the second query phase. 
    After the challenger sent all public variables to the adversary, the adversary starts the first query phase by requesting anything it wants but eventually, it has to send a, so-called, \textit{challenge tuple} to the challenger.
    This tuple is of the form \(\goedel{X_i, X_j, M}\) and represents the info the adversary wants to be challenged to.
    Then, the challenger flips a coin \(b\in\Set{0,1}\) and if \(b = 0\) it sends an ambiguous signature \(\sigma\) to the adversary on behalf of \(X_i\).
    Otherwise, the challenger sends an ambiguous signature on behalf of \(X_j\). Now, the adversary can start its second query phase, and eventually sends the challenge bit, which is a guess whose signature the challenger sent to our adversary.
    The adversary wins if the challenge bit matches the flipped coin of the challenger because then it managed to circumvent the ambiguity of the scheme.
  
    \subsection{Unforgeability}\label{unforgeability}
      We, again, consider the previously defined adversary--challenger game.
      The adversary can query everything it wants, but finally it has to return \(S=\goedel{\sigma, X_c, X_d, M}\) where \(\sigma=\goedel{s, h_1, f}\), \(X_c, X_d\) being public keys and \(M\in\mespace\) being the message.
      Continuing, the adversary wins the game if \(\textbf{AVERIFY}(S) = accept\) and if either
        \begin{enumerate}
          \item No \textbf{ASign} query was made with input \(\goedel{X_c, X_d, f, M}\) or \(\goedel{X_d, X_c, h_1, M}\).
                And no \textbf{Private Key Extraction} query was made on either \(X_c\) or \(X_d\).
          \item No \textbf{ASign} query was made with input \(\goedel{X_c, X_i, f, M}\) for all \(X_i \neq X_c, X_i\in\uspace\) and no \textbf{Private Key Extraction} query was made for \(X_c\).
        \end{enumerate}