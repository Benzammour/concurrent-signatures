For our following construction and the proof, later on, we will need certain constructs and tools.
These will be defined in the following.

\begin{definition}[\textnormal{\textit{Random Oracle}}]
  A \textit{random oracle} is a black-box algorithm which serves each \textit{unique} request with a truly random value. Moreover, it serves each duplicate request with the same answer it sent before.
\end{definition}

\begin{definition}[\textnormal{\textit{Chosen Message Attack}}]
  In a signature scheme, a \textit{chosen message attack} enables the adversary to get the signature of several messages, meaning it can query the signing oracle.
\end{definition}

\subsection{Mathematical Concepts}
  This chapter is going to introduce all mathematical concepts needed to follow the line of thought in the coming course of action. 
  
  \begin{definition}[\textnormal{\textit{Generator}}]
    Let \(\mathbb{G}\) be a cyclic group of order \(p\). Then a group generator is \(g\in\mathbb{G}\) such that it fulfills
      \[\mathbb{G} = \Set{g^i \bmod p \mid i\in\nats}\]
    This means, that by taking the generator to multiple powers, while taking the modulus of \(p\), we can generate all elements of our given group.
  \end{definition}

  \begin{definition}[\textnormal{\textit{Discrete Logarithm}}]
    In a group \(\mathbb{G}\), we define the \textit{discrete logarithm} as an integer \(x\) such that \(b^x = a\) for an arbitrary integer \(a\) and \(b^x = \overbrace{b\cdot \ldots\cdot b}^{x \text{ times}}\).
  \end{definition}

  \begin{definition}[\textnormal{\textit{Negligible function}}]
    We say a function \(f\) is negligible if for all \(c\in\nats^{\geq 1}\) there exists a \(n_0\in\nats\) such that for all \(n \geq n_0\)
    \[f(n) < \frac{1}{n^c}\]
    holds. This means, that our function \(f\) converges towards 0. 
  \end{definition}

\subsection{Hardness of the Discrete Logarithm}
  We define the hardness of the discrete logarithm analogously to \cite{katz2014introduction}, starting with the

  \begin{definition}[\textnormal{\textit{Discrete logarithm experiment}}]
    For an experiment \(DLog_{\mathcal{A}, \mathcal{G}}(\ell)\) we consider \(\mathcal{G}\) to be a cyclic group generator, and \(\mathcal{A}\) to be a polynomial-time algorithm.
    Firstly, we start off by running \(\mathcal{G}(1^\ell)\) to acquire \((\mathbb{G}, p, g)\) where $\mathbb{G}$ is a cyclic group of order \(p\), and \(g\in\mathbb{G}\) is a group generator of \(\mathbb{G}\).
    We pick an \(a\in\mathbb{G}\) at random, and pass all information to an algorithm \(\mathcal{A}\).
    If \(\mathcal{A}\) is able to calculate an \(x\) such that \(g^x = a\) we return 1 and 0 otherwise.
  \end{definition}

  Now we define the actual hardness property we want to acquire

  \begin{definition}
    We say that the \textit{discrete logarithm problem is hard} relative to \(\mathcal{G}\) if for all algorithms \(\mathcal{A}\) there exists a negligible function \(\mathrm{negl}\) such that
      \[\mathrm{Pr}[DLog_{\mathcal{A}, \mathcal{G}}(\ell) = 1] \leq \mathrm{negl}(\ell)\]
  \end{definition}

  That means, that it is hard for an algorithm to reliably calculate the discrete logarithm of any number \(a^\prime\in\mathbb{G}\).

\subsection{Forking Lemma}
  In \cite{pcstern96} and \cite{pointcheval2000security}  Pointcheval and Stern introduced a tool for signature schemes, the \textit{forking lemma}. 
  It states that if an algorithm \(E\) can produce a signature \(\goedel{r_1, h, r_2}\) on public input, where \(r_1\) randomly chosen from a huge set and \(h\) being the hash of the message and \(r_1\).
  Also \(r_2\) only depending on \(r_1, h\) and the message.
  Then there exists another algorithm \(\mathcal{A}\) that controls said algorithm \(E\).
  \(\mathcal{A}\) can then replay \(E\), replacing the signer by the simulation, to generate a second valid signature \(\goedel{r_1, h', r_2'}\) with \(h\neq h'\).

\subsection{Notation}
  We define a certain set of notation to make communicating concepts easier.
  \begin{itemize}
    \item We define \(\cdot \| \cdot \) as string concatenation
    \item We define \(k\) to be be the keystone
  \end{itemize}