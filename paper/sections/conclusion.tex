To conclude, in this paper we illustrated the concept of concurrent signatures defined by Chen et al., and provided a rundown of the unforgeability proof provided in \cite{chen2004concurrent}.

In short, the concurrent signature scheme enables parties to sign a contract ambiguously, in respect to their own identity, until another piece of information is released.
This piece of information is called the keystone and is generated by one of the participating parties.
To add, the generation of the keystone by one of the participating parties is also the reason why the concurrent signature scheme does not solve the problem with the \textit{truly fair} signature exchange.
However, the presented solution is an adequate enough solution for most applications.

Lastly, keep in mind that the unforgeability property is not a sufficient condition for the security of this scheme.
To see how the other two properties are proven, reference \cite{chen2004concurrent}, or if you are interested on how this scheme is making use of its underlying signature structure, see \cite{abe20021} and \cite{schnorr1991efficient}.