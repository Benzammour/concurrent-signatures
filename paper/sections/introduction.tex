Creating a signature scheme that provides true fairness in exchanging signatures is a well-studied problem and a few approaches already exist trying to solve it.

Among them is, for example, the optimistic fair exchange presented in \cite{asokan1998optimistic} with the usage of a trusted third party that can supervise the signature exchange and rule out any fraud on either side. 
The problem with this would be the need of a third entity, even if it is only needed when one party wants to cheat the other. 

The concurrent signature scheme \cite{chen2004concurrent} tries to solve that by using random bits to obfuscate the signing parties, who will be ambiguous to every other party.
With this, neither party is at a disadvantage, e.g. if the other party does not want to sign the item anymore.
These signatures are bound to their respective signer by releasing the key that generates said random bits, we call this key the \textit{keystone}.
This concurrent scheme does not provide a solution for a \textit{truly fair} signature exchange, however, it provides an adequate solution for most applications.

In this paper, we will show how such a concurrent signature scheme can be constructed.
The focus of this paper, however, lies in providing an intuitive understanding of the unforgeability proof provided in \cite{chen2004concurrent}.
We will make use of a concrete scheme defined in \cite{chen2004concurrent} which builds upon other signature schemes.
Note, that this signature scheme is not limited to the concrete one we define here, each suitable scheme can be used to produce a concurrent protocol.

In the following chapter, we will define certain preliminaries that we are going to need throughout this paper.
\Cref{construction} defines the concrete construction of this scheme and \cref{protocol} describes how our construction should be used.
\Cref{secmodel} defines when our signature scheme can be labeled as \textit{secure} and, lastly, \cref{unforgproof} explains the unforgeability proof of \cite{chen2004concurrent}.