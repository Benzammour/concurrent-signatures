Creating a signature scheme that provides true fairness in exchanging these signatures is a well-studied problem and a few approaches already exist trying to solve it.

Among them is, for example, the optimistic fair exchange presented in \cite{asokan1998optimistic} with the usage of a trusted third party that can mitigate the signature exchange and rule out any fraud on either side. 
The problem with this would be the need of a third entity, even if it is only needed when one party wants to cheat the other. 

The concurrent signature scheme \cite{chen2004concurrent} tries to solve that by using random bits to obfuscate the signing parties, who will be ambiguous to every other party.
With this, neither party is at a disadvantage, e.g. if the other party does not want to sign the item anymore.
These signatures are bound to their respective signer by releasing the key to generating these random bits.

In this paper, we will show how such a concurrent signature scheme can be constructed.
However, the focus of this paper lies in providing an intuitive understanding of the unforgeability proof provided in \cite{chen2004concurrent}.

In the following chapter, we will define certain preliminaries that we are going to use throughout this document.
\Cref{construction} defines the concrete construction of this scheme, \cref{secmodel} defines when our signature scheme can be labeled as \textit{secure}.
Lastly, \cref{unforgproof} provides the unforgeability proof of \cite{chen2004concurrent} in a clear matter.