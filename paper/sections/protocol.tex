This section is responsible for providing an understanding of the concrete signature protocol and how the algorithms defined in \autoref{construction} should be used.

We will illustrate the protocol through an exemplary interaction between a party A(lice) and B(ob). 
The following description will be visually supported through \autoref{fig:concprotocol}. 

Firstly, we start by having both parties run the \textbf{SETUP} function to initialize all necessary variables for this scheme.
Our, so called, \textit{initial signer} continues with generating a keystone \(k\in\kspace\) alongside the corresponding keystone-fix \(f\in\kfspace\).
Keep in mind, that the keystone-fix is generated through \(\textbf{KGEN}(k)\).
We set, without loss of generality, Alice as our initial signer.
Continuing, the initial signer starts off by signing a message \(M_A\in\mespace\) through \textbf{ASIGN}, while providing the private keys of herself and Bob, her own private key \(x_A\) and the keystone-fix \(f\).
Afterwards, Alice sends the resulting ambiguous signature \(\sigma_A = \goedel{s_A, h_A, f}\) to Bob.
He verifies said signature by checking if \textbf{AVERIFY} returns \(accept\), for which he additionally provides their respective public keys, as well as, Alice's message \(M_A\).
Note, that we are not trying to encrypt or obfuscate the messages in any way.
Bob and Alice, repeat the signing and verification steps analogously for Bob with the \textit{same keystone-fix}, i.e. Bob generates and sends his ambiguous signature with \(f\) and Alice verifies it.
Lastly, if Alice is satisfied with Bobs signature, she publishes the keystone \(k\) with which both \textit{ambiguous signatures} will be binding for both parties.

The power of who is generating and holding the keystone-fix is the one thing that keeps the concurrent signature scheme to be \textit{truly fair}.
Because with this, the initial signer (i.e. the one that generates the keystone(-fix)) can withhold the keystone.
However, this would not be beneficial for the withholding party in real applications.
Moreover, even if the withholding party keeps the keystone, without releasing it, the other party is not bound to the signature, meaning it is at no disadvantage. 

\begin{figure}[htbp]
  \centering
  
  \resizebox{.99\hsize}{!}{
    \begin{tikzpicture}[node distance=2cm,auto,>=stealth']
      \node[] (bob) {B};
      \node[left = of bob] (alice) {A};
      \node[below of=bob, node distance=8cm] (bobground) {};
      \node[below of=alice, node distance=8cm] (aliceground) {};
      
      % vertical lines
      \draw (alice) -- (aliceground);
      \draw (bob) -- (bobground);
    
      % setup
      \node[left, align=left] at ($(alice)!0.1!(aliceground)$) {\textbf{SETUP}};
      \node[right, align=right] at ($(bob)!0.1!(bobground)$) {~\textbf{SETUP}};
  
      % k keystone + KGEN
      \node[left, align=left] at ($(alice)!0.225!(aliceground)$) {$k\in\kspace$, $f = \textbf{KGEN}(k)$};
      
      % ASIGN + Arrow
      \node[left, align=left] at ($(alice)!0.4!(aliceground)$) {$\textbf{ASIGN}(X_A, X_B, x_A, f, M_A)$ \\ {$= \goedel{s_A, h_A, f} \eqqcolon \sigma_A$}};
      \draw[->, dashed] ($(alice)!0.4!(aliceground)$) -- node[sloped, above]{$\sigma_A$} ($(bob)!0.45!(bobground)$);
      
      % AVERIFY
      \node[right, align=center] at ($(bob)!0.45!(bobground)$) {$\textbf{AVERIFY}(\sigma_A, X_A, X_B, M_A)$};

      % ASIGN + Arrow
      \node[right, align=left] at ($(bob)!0.65!(bobground)$) {$\textbf{ASIGN}(X_B, X_A, x_B, f, M_B)$};
      \draw[->, dashed] ($(bob)!0.65!(bobground)$) -- node[sloped, above]{$\sigma_B$} ($(alice)!0.7!(aliceground)$);

      % AVERIFY
      \node[left, align=left] at ($(alice)!0.7!(aliceground)$) {$\textbf{AVERIFY}(\sigma_B, X_B, X_A, M_B)$};

      % keystone
      \draw[->, dashed] ($(alice)!0.8!(aliceground)$) -- node[above]{$k$} ($(bob)!0.85!(bobground)$);
    \end{tikzpicture}
  }

  \caption{Concurrent Signature Protocol - An Overview}
  \label{fig:concprotocol}
\end{figure}