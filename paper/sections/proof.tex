To start, we are going to describe the idea of the proof, which will dictate our further course of action

\begin{enumerate}[1.)]
  \item \textit{Assumption:} The discrete logarithm problem is hard.
  \item \textit{Proof by contraposition:} We assume our scheme is not unforgeable.
  \item We construct an algorithm that leverages the forgeability of our scheme to contradict our assumption. 
\end{enumerate}
%
If our signature scheme is \textit{not} unforgeable, there exists an algorithm \(E\) that can forge a signature with non-negligible probability.
Moreover, we define our algorithm \(\mathcal{A}\) that solves the discrete logarithm as follows:
\(\mathcal{A}\) takes \(\goedel{g,X,p,q}\) as its input and uses \(E\) as a component to solve the discrete logarithm problem, which would result in a contradiction to our assumption.

Before we start constructing \(\mathcal{A}\) we need to show that we can rewrite our ambiguous signature format \(\sigma = \goedel{s,h_1, f}\) as \(\goedel{r_1, h, r_2}\) in order to use the forking lemma later on.
We define \(r_1\) to be \(g^t X_j^f \bmod p\), which fulfills the property that it takes its values randomly from a huge set \(\ints_q\), \(h = h_1 + f\) is the hash of the message \(M\) and \(r_1\), and lastly \(r_2 = s\) which depends solely on \(r_1, h\) and \(M\).

Furthermore, we model our hash functions \(H_1, H_2\) as random oracles, In our following construction the algorithm \(\mathcal{A}\) simulates these random oracles and the challenger \(C\) that our forging-adversary \(E\) plays against.

Now, our algorithm \(\mathcal{A}\) starts by generating our participants \(\mathcal{U}\). It guesses that our forging-adversary \(E\) chooses \(X_\alpha\) for the position \(X_c\) in its output, and, therefore, sets \(X_\alpha = X\).
This means, that \(\mathcal{A}\) wants to acquire the discrete logarithm for the challenge public key \(X_\alpha\).
It also chooses a private key \(x_i\), which is randomly chosen from \(\ints_q\), for all participants \(i\) that are not \(\alpha\).
Otherwise, the game would be pointless due to generating the discrete logarithm, we want to acquire, on our own.
After all of that, \(\mathcal{A}\) passes on \(\goedel{g,p,q}\), as well as the public keys, to our forging-adversary.
As an overview of our setup, for now we have 

\begin{figure}[htbp]

  \begin{center}
    \resizebox{\hsize}{!}{
      \begin{tikzpicture}[custom/.style = {rectangle, draw, align=center, minimum width=1.5cm, inner xsep=3mm, inner ysep=3mm}, -latex]
        % nodes
        \node[custom] (adversary) {$E$};
        \node[above=.75cm of adversary] (adversary_input) {$\goedel{g, p, q}$};
  
        \node[custom, right=.75cm of adversary_input] (participants) {$\mathcal{U}$};
        \node[custom, right=.75cm of participants] (priv_keys) {$x_i$};
        
        \node[custom, right=.65cm of adversary] (pub_keys) {$X_i = g^{x_i} \bmod p$};
        
        \node[fit={(adversary_input) (priv_keys) (pub_keys)}, draw, outer ysep=2mm, inner xsep=5mm, inner ysep=5mm, label={\Large$\mathcal{A}$}] (outer_box) {};
        \node[left=.5cm of outer_box.west] (input) {$\goedel{g, X, p, q}$};
        
        % connections
        \draw (input.east) to (outer_box.west);
        \draw (adversary_input.south) to (adversary.north);
        \draw (pub_keys.west) to (adversary.east);
      \end{tikzpicture}
    }
  \end{center}

  \caption{Unforgeability Proof - Overview}
  \label{fig:proofoverview}
\end{figure}

Now, our forging-adversary \(E\) can start querying requests, which include all of the queries from \autoref{secmodel} with a few additions and changes.
We add \(H_1\)- and \(H_2\)-queries, with which \(E\) can query the respective oracle to acquire a hash. 
Moreover, we modify our \textbf{ASign} queries such that if \(\mathcal{A}\) receives a correct input \(\goedel{X_i, X_j, f, M}\) with \(i\neq\alpha\) we return a signature as defined in \autoref{secmodel}.
However, if \(i = \alpha\), \(\mathcal{A}\) picks random values for \(h_1\) and \(s\), and calculates the \(H_2\) hash from these values.
This happens with the premise that the random values and resulting string have never appeared in any \(H_2\) query before, otherwise \(\mathcal{A}\) chooses other values for \(h_1\) and \(s\) and starts again.

After a number of requests our forging-adversary returns an ambiguous signature \(\sigma = \goedel{s,h_1, f}\), public keys \(X_c\) and \(X_d\), and the message \(M\).
We know, from our definition in \autoref{unforgeability}, that the adversary wins if \textbf{AVERIFY} accepts \(\sigma\) alongside the public keys and the message, and if one of the following cases holds
  \begin{enumerate}
    \item No \textbf{ASign} query was made with input \(\goedel{X_c, X_d, f, M}\) or \(\goedel{X_d, X_c, h_1, M}\).
          Also, no \textbf{Private Key Extraction} query was made on either \(X_c\) or \(X_d\).
    \item No \textbf{ASign} query was made with input \(\goedel{X_c, X_i, f, M}\) for all \(X_i \neq X_c, X_i\in\uspace\) and no \textbf{Private Key Extraction} query was made for \(X_c\).
  \end{enumerate}

In \cite{abe20021} Abe et al. showed that our first case is negligible, assuming the hardness of the discrete logarithm problem. 
We know that our forging-adversary forges signatures with non-negligible probability, therefore, the second case must have occurred.

Furthermore, we assume that \(X_c = X_\alpha = X\) because otherwise, \(\mathcal{A}\) aborts as a result of failing to predict the correct the challenge public key.
% \textcolor{blue}{explain further?} 

Because \textbf{AVERIFY} returns \(accept\) and we know that it checks \(h_1 + f \overset{?}{=} H_2(g^s X_{c}^{h_1} X_d^f \bmod p ~\|~ M)\), we now have the equation
\[h_1 + f \overset{?}{=} H_2(g^s X_{c}^{h_1} X_d^f \bmod p ~\|~ M)\]
which leaves two new cases to investigate.

In the \textit{first} case \(h = h_1 + f\) has never occurred in any signature query before.
If that is the case, we rewrite our ambiguous signature to \(\goedel{r_1, h, r_2}\) and, by using the forking lemma, force \(E\) to produce a second signature \(\goedel{r_1, h', r_2'}\) with \(h\neq h'\).

We now have two ambiguous signatures (by rewriting the second one) with \(\sigma = \goedel{s, h_1, f}\) and \(\sigma^\prime = \goedel{s', h_1', f'}\), and \(h \neq h'\).
This results in \(h = h_1 + f \neq h_1' + f'\), which implies that either \(h_1 \neq h_1'\) or \(f \neq f'\).
The case in which \(h_1 = h_1'\) is negligible because then the relevant \(h_1\) queries would have been calculated before the \(H_2\) queries that produced \(h\) and \(h'\) which is not possible.
In addition, it would imply that \(f = f'\) but if that would be the case the output of our \(H_1\) (that produces \(f\) and \(f'\)) oracle would have to match an \(H_2\) query.
Henceforth, we know that \(h_1 \neq h_1'\) and \(f = f'\).
As a consequence of \(h\) and \(h'\) resulting from different oracle queries we have
\begin{align}
  g^{s} X^{h_1} X_d^f &= g^{s'} X^{h_1'} X_d^{f'} \\
  s + x h_1 + f &= s' + x h_1' + f' \quad\mid f=f'\\
  s + x h_1 &= s' + x h_1' \qquad \mid h_1 \neq h_1' \label{divbyh1}\\
  x &= \frac{s-s'}{h_1' - h_1}
\end{align}
So in the first case, \(\mathcal{A}\) solves the discrete logarithm problem of \(X\) with non-negligible probability.
This contradicts the hardness of the discrete logarithm problem and therefore, for the first case, this scheme is unforgeable.

Continuing with the second case, here we assume that \(h = h'\) with \(h' = h_1' + f\) appeared in a previous signature query.
We denote said signature request with \(\goedel{X_{c'}, X_{d'}, f', M'}\) and its resulting signature with \(\sigma' = \goedel{s', h_1', f'}\).
Because we have \(h = h'\) we know that
  \[g^s X^{h_1} X_d^f \bmod p  = g^{s'} X_{c'}^{h_1'} X_{d'}^{f'} \bmod p \]
and \(M = M'\) have to hold.
Otherwise, the input of the hash functions would not be the same, and therefore \(h \neq h'\).

At this point we have to differentiate further.
If \(X_{c'} = X\) or \(X_{d'} = X\) but their exponents differ, or \(X_{c'},X_{d'} \neq X\) we can easily derive the discrete logarithm \(x\) of \(X\) with the same argument as in the first case.
It is important for the case where \(X_{c'} = X\) or \(X_{d'} = X\), that the exponents differ because otherwise we would have a division by 0 in step \ref{divbyh1} of resolving to \(x\).

However, let us take a closer look at the case where \(X_{c'} = X\) or \(X_{d'} = X\), and their exponents are \textit{the same}.
If we had \(X_{c'} = X\) and \(h_1 = h_1'\), because the exponents (i.e. \(h_1, h_1'\)) are the same, then that would imply \(f = f'\) due to \(h = h'\).
This, however, is a contradiction to the condition that no \textbf{ASign} queries can be made of the form \(\goedel{X_c, X_i, f, M}\) because we set \(X_c = X\).
Now, what if we had \(X_{d'} = X\) and \(h_1 = f'\).
Then, when \(h_1' = h' - f'\) was generated in a previous \textbf{ASign} query, \(f = h_1'\) already had to hold by definition.
Moreover, we also know that the output of \textbf{ASign} queries is determined by oracles \(H_1\) and \(H_2\).
Now, the problem is that \(f\) also is generated by \(H_1\). 
Meaning that an output of \(H_1\) has to match an \(h_1'\), that is generated by \(H_2\) and \(H_1\).
The probability of this occurring is negligible and therefore not relevant. 

With this, we showed that the discrete logarithm problem could also be solved in the second case.
This leads to a contradiction and thus the \cref{unforgeablelemma} holds.
